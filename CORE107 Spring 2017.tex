

\documentclass[10pt,preprint]{aastex}

\usepackage{indentfirst}
\usepackage{hanging} 
\usepackage{verbatim}
\usepackage{graphicx}
\usepackage{multirow}
\usepackage[english]{babel} 

\usepackage{csquotes}
\MakeOuterQuote{"}

\topmargin 0in 
\oddsidemargin 0in 
\evensidemargin 0in 
\textwidth 6.5in 
\parindent 0pt

\pagestyle{plain}
\renewcommand{\thefootnote}{\fnsymbol{footnote}} 

\begin{document}

\setcounter{page}{1} \pagenumbering{arabic}

{\bf Exploration and Discovery II\\ 
Politics and the 1960's:  Turmoil and Transformation\\
Core 107 Section 24\\
Spring Semester 2017\\ 
Lewis \& Clark College\\ \\
}
\setlength{\unitlength}{1in} 
\begin{picture}(1,0)
\setlength{\unitlength}{2.5cm} 
\linethickness{4 pt} \put(0,.1) {\line(1,0){6.5}} 
\end{picture} 
\vspace*{.15in}

\renewcommand{\arraystretch}{1} 
\begin{tabular}{lp{5.25in}}
\textbf{Professor:}\dotfill & \dotfill Ian McDonald, Ph.D.\\
\textbf{Meets:}& \dotfill Monday, Wednesday, and Friday\\ 
\textbf{}& \dotfill 1:50pm - 2:50pm\\ 
\textbf{Classroom:}&\dotfill JR Howard 133\\ 
\textbf{Office Hours:}& \dotfill  JR Howard 329\\ 
\textbf{}& \dotfill MW 11:30am - 1:30pm and by appointment\\ 
\textbf{Email:}&\dotfill imcdonald@lclark.edu\\ \end{tabular}

\vspace*{.15in} \noindent 
\textbf{Overview and Course Goals}\\

\noindent In the 1960's, the American political scene was filled with tension and high drama.  Does the legacy of this period define our political lives today?  

In this class, we will examine how the events of the 1960’s endure in the contemporary political scene.  Political polarization, dominance of a nationalized political agenda, outsized expectations of the U.S. presidency, and decline of trust in institutions all have roots in this period.  We will give emphasis to the civil rights movement, Vietnam, student uprisings, the rise of the modern conservatism, emergence of nationalized media, and the interactions of these developments. 

Studying this period will help us critically assess a larger question:  does the past reveal the future?  Is our current historical moment recognizable in the conflict of the 1960's?  Could we have predicted the events of 2016 and 2017 if we had read the signs?  Historians like to think that lessons of the past inform our understanding of the present.  Are they right?  

Note that we will discuss several films in class, and viewing these films in advance will be assigned.  

\textbf{Participation}:  Core 107 is not a lecture-based course; its success and yours will depend upon your active, informed participation in class discussions and group work. Because the readings are the foundation of the course, you must come to each class having read and thought about the required material and be prepared to participate in class discussion.  In addition, you must come prepared to discuss the films on the designated days.  This means you must watch the films completely and study them carefully before class. Your participation is a requirement of the course.  In-class note taking is likewise mandatory.

\textbf{Film Screenings}:  The required films in the class should be available online (YouTube in some cases), but there will be evening group screenings at dates to be announced.

\textbf{Attendance}: Regular attendance at class sessions, at lectures, and at films designated in the syllabus is essential for successful completion of this course. Attendance will be considered in the assignment of final grades. Should you need to miss class for medical reasons, it is your responsibility to notify me as soon as possible and to arrange to make up any missed work.  Should you ever have a medical or other condition that results in the need for extended absence, you must both speak with me and provide appropriate documentation.  The Student Support Services Office can assist you in the event of necessary prolonged absences.

\textbf{Laptops and Electronic Devices}:
Laptops, cell phones, and other electronic devices may not be used during class without the permission of the instructor. You should therefore make sure to print all readings before class. This policy is motivated by the \href{http://www.sciencedirect.com/science/article/pii/S0360131512002254}{growing body of research} which finds that the use of laptops hinders learning not just for the people who use them but the students around them as well. Multitasking is unfortunately distracting and cognitively taxing. In addition, research suggests that students take notes more effectively in longhand than when they write on laptops.

\textbf{Papers}:  There are three formal writing assignments: two 5-pg. essays focusing on course materials, and one 10- to 12-pg. paper that will include sources beyond the syllabus.  The first paper is due \textbf{February 8 at 10pm} and the second is due \textbf{March 8 at 10pm}.  More specific instructions regarding topics and resources to be used will be made available in class.

\textbf{Oral Presentation}:  Each student will discuss her work on the 10- to 12-pg. paper (Paper 3) with the class during time set aside toward the end of the semester.  Classmates will be responsible for providing feedback to the presenter concerning the content of the presentation.  Details about the character of these presentations will be provided in due course.

\textbf{Festival of Scholars}:  The Festival of Scholars is a campus-wide celebration of student work. It is an opportunity to discuss research, to exhibit, perform, or appreciate art, and to cross disciplinary boundaries. \textbf{The festival will be held on April 14.} Classes will be cancelled on that day, but you are still required to participate in the Festival, either by presenting your work or attending presentations by your fellow students. When the final program for the Festival becomes available, I will recommend attendance of specific presentations, and will explain how attendance will contribute to your course grade.

\textbf{Late Work}:  All assignments are due on the schedule dates. As a general rule, late work will only be accepted with a 2/3 grade penalty per day, and will not be accepted after two days. In unavoidable circumstances, such as illness, you have the responsibility to contact me as soon as possible to make arrangements for timely completion of assignments.  

\textbf{Incomplete Work}: Students must complete ALL of the assigned work (i.e., every single assignment) in order to receive any credit for the course, unless there are extenuating circumstances.  You must confer with me in advance of any situation that might lead to late or incomplete work.

\textbf{The Final-Exam Date}: The Final Exam must be taken on the official assigned date.  Do not make travel arrangements that will require you to miss the exam.

\textbf{Learning Disabilities}: If you have a disability that may impact your academic performance, you may request accommodations by submitting documentation to the Student Support Services Office in the Albany Quadrangle (x7156).  After you have submitted documentation and filled out paperwork there for the current semester requesting accommodations, staff in that office will notify me of the accommodations for which you are eligible. If you have a disability that may impact your academic performance, you may request accommodations by submitting documentation to the Student Support Services Office in the Albany Quadrangle (x7156).  After you have submitted documentation and filled out paperwork there for the current semester requesting accommodations, staff in that office will notify me of the accommodations for which you are eligible.

\textbf{Academic Integrity}: I expect that any work you submit in this course will be your own and that you will cite any sources you have used. Failure to do so can be plagiarism, a serious academic offense that can result in your suspension or even expulsion from the college. I expect you to understand and abide by the College’s Academic Integrity Policy and Procedures.  If you have any questions about the policy, I encourage you to come and talk with me as soon as possible. 

\textbf{Changing Class Sections or Withdrawing}: Any requests to change E\&D sections should be directed to the E\&D Program Coordinator, Dawn Wilson (Miller 404, explore@lclark.edu); changes to course section will not be permitted after the third class session.  Because it is a required course designed for first year students, withdrawal from E\&D is not permitted (unless a student is withdrawing from all courses for the semester).

\textbf{Campus Resources}: Because you are new to the college and because of the nature of the course, please consider using some of the many helpful resources available to you on campus: Information Technology, Watzek Library, International Student Services, SAAB Tutoring, and (especially important) the Writing Center.  To maximize your benefit from these resources, you need to plan ahead and to be active and professional when consulting them.  When appropriate, tell me about how you are using these resources to advance your performance in the course.  I may be able to enhance your use of the resource, and I am also happy to consider your work as part of your effort at course participation.

\textbf{Statement of Learning Objectives}

1.	You will develop your ability to work collaboratively and constructively in a group setting.  In particular, you will demonstrate capacity to listen, respond reasonably to, and build on the intellectual positions staked out by course materials, instructors, and fellow students.

2.	You will compose effective formal academic writing, both in class (during exams or other exercises) and outside of class (for take-home writing assignments).  This means you will demonstrate capacity to compose clear and compelling prose, formulate persuasive arguments supported with evidence, and pursue creative and eloquent expression of ideas.

3.	You will pursue original research leading to the production of a major essay, in which students will use appropriate research tools and sources to meet the information need.
You will also demonstrate capacity to formulate a compelling research question, employ various research sources and methods, formulate an original thesis supported with sufficient evidence, and produce proper citations and reference lists.

4.	You will develop your ability to prepare and deliver a formal oral presentation that demonstrates a capacity to orally present original ideas in a coherent and persuasive manner, and respond to audience questions and remarks)\\
\\


\renewcommand{\arraystretch}{1} 

\begin{tabular}{lp{2.5in}}
\textbf{Grading formula}\\
\\
Writing, including research project:& \dotfill 50\%\\			
Midterm: & \dotfill 10\%\\		
Graded oral presentation & \dotfill 10\%\\
Participation: & \dotfill 10\%\\
Final Exam: & \dotfill 20\%\\ 
\end{tabular}
\vspace*{.15in} \noindent 
		
\begin{deluxetable}{ll} 

\tabletypesize{\footnotesize}
\tablecolumns{2} 
\tablewidth{0pt} 
\tablecaption{Topics and Reading Assignments} 
\tablehead{\colhead{Date} & \colhead{Topic}                                                                   }

\startdata

\textbf{1:}
\textbf{Jan 18 (Wed)}&
\textbf{Introduction}\\
&Exerpts from \textit{Mad Men}:  Smoke Gets In Your Eyes (Season 1 Episode 1)\\
\\

\textbf{2:}
\textbf{Jan 20 (Fri)}&
\textbf{Introduction}\\ 
&Orwell, \textit{Politics and the English Language}\\
&Assignment:  Find one piece of contemporary writing, or your own, that demonstrates problems that Orwell describes.\\
\\

\textbf{3:}
\textbf{Jan 23 (Mon)}&
\textbf{The Civil Rights Movement:  Overview}\\
&Perlstein, \textit{Before the Storm}, c.1-4\\
\\

\textbf{4:}
\textbf{Jan 25 (Wed)}&
\textbf{The Civil Rights Movement:  Overview}\\
&Valelly, \textit{The Two Reconstructions} (online) c.6\\
&Hofstadter, \textit{Paranoid Style of America Politics} (online)\\
\\

\textbf{5:}
\textbf{Jan 27 (Fri)}&
\textbf{The Civil Rights Act of 1964}\\
&Purdum, \textit{An Idea Whose Time Has Come}, c. 1 and 2\\
\\

\textbf{6:}
\textbf{Jan 30 (Mon)}&
\textbf{The Civil Rights Act of 1964}\\
&Purdum, \textit{An Idea Whose Time Has Come}, c. 3 and 4\\
\\

\textbf{7:}
\textbf{Feb 1 (Wed)}&
\textbf{The Civil Rights Act of 1964}\\
&Read Title II, III, and IV of the Civil Rights Act and\\
&King, \textit{Letter from a Birmingham Jail}\\
\\

\textbf{8:}
\textbf{Feb 3 (Fri)}&
\textbf{The Civil Rights Act of 1964}\\
&Purdum, \textit{An Idea Whose Time Has Come}, c. 5 and 6\\
\\

\textbf{9:}
\textbf{Feb 6 (Mon)}&
\textbf{The Civil Rights Act of 1964}\\
&Purdum, \textit{An Idea Whose Time Has Come}, c. 7 and 8\\
\\

\textbf{10:}
\textbf{Feb 8 (Wed)}&
\textbf{The Civil Rights Act of 1964}\\
&Perlstein, \textit{Before the Storm} c.5-8\\
&\textbf{Paper Assignment 1 Due in Moodle at 10pm}\\
\\

\textbf{11:}
\textbf{Feb 10 (Fri)}&
\textbf{The Civil Rights Act of 1964}\\
&Purdum, \textit{An Idea Whose Time Has Come}, c. 9, 10\\
&\textbf{Topic to Pursue assignment due}\\
\\

\textbf{12:}
\textbf{Feb 13 (Mon)}&
\textbf{The Reaction to Civil Rights}\\
&Perlstein, \textit{Before the Storm} c. 9-11\\
\\

\textbf{13:}
\textbf{Feb 15 (Wed)}&
\textbf{The Reaction to Civil Rights}\\
&Purdum, \textit{An Idea Whose Time Has Come}, c. 11, 12, and Epilogue\\
\\

\textbf{14:}
\textbf{Feb 17 (Fri)}&
\textbf{The Reaction to Civil Rights}\\
&Perlstein, \textit{Before the Storm} c. 12-14\\
\\

\textbf{15:}
\textbf{Feb 20 (Mon)}&
\textbf{The U.S. Presidency in the Public Imagination}\\
&Perlstein, \textit{Before the Storm} c. 15-16 \\
&\textbf{Annotated Working Bibliography Assignment I due}\\
\\

\textbf{16:}
\textbf{Feb 22 (Wed)}&
\textbf{The U.S. Presidency in the Public Imagination}\\
&Zelizer, The Power of Lyndon Johnson is a Myth \textit{Washington Post} Jan 11, 2015 (online)\\
&Perlstein, \textit{Before the Storm}, c. 17-18\\
\\

\textbf{17:}
\textbf{Feb 24 (Fri)}&
\textbf{The U.S. Presidency in the Public Imagination}\\
&Film Discussion:  \textit{Dr. Strangelove}\\
\\

\textbf{18:}
\textbf{Feb 27 (Mon)}&
\textbf{The Vietnam War, and the Nationalization of American Politics:  Part I}\\
&O'Brien:  \textit{The Things They Carried}:  How to Tell a True War Story, and The Man I Killed\\
\\

\textbf{19:}
\textbf{Mar 1 (Wed)}&
%\textbf{The Vietnam War, and the Nationalization of American Politics:  Part I}\\
%&Mailer:  Part I:  Nixon in Miami\\
\textbf{Overview of Research Project}\\
\\

\textbf{20:}
\textbf{Mar 3 (Fri)}&
\textbf{The Vietnam War, and the Nationalization of American Politics:  Part I}\\
&Mailer:  Part II:  The Siege of Chicago\\
&\textbf{Essay 1 due}\\
\\

\textbf{21:}
\textbf{Mar 6 (Mon)}&
\textbf{The Vietnam War, and the Nationalization of American Politics:  Part II}\\
&O'Brien, \textit{The Things They Carried}, The Things They Carried (title chapter)\\
\\

\textbf{22:}
\textbf{Mar 8 (Wed)}&
%\textbf{The Vietnam War, and the Nationalization of American Politics:  Part I}\\
%&Hallin, \textit{The Uncensored War}, c. 4 and 5\\
\textbf{Tenatively Scheduled Library Session:  Meet in Watzek Classroom}\\
&\textbf{Paper Assignment 2 Due in Moodle at 10pm}\\
\\

\textbf{23:}
\textbf{Mar 10 (Fri)}&
\textbf{The Vietnam War, and the Nationalization of American Politics:  Part III}\\
%&\textit{The Thintbegs They Carried}, TBD\\
&O'Brien, \textit{The Things They Carried}, The Ghost Soldier\\
\\

\textbf{24:}
\textbf{Mar 13 (Mon)}&
\textbf{Where Did the Grownups Go?  Rebellion and the Generation Gap}\\
&Didion, Slouching Toward Bethlehem\\
\\

\textbf{25:}
\textbf{Mar 15 (Wed)}&
\textbf{Where Did the Grownups Go?  Rebellion and the Generation Gap}\\
&Young Americans for Freedom, \textit{The Sharon Statement}\\
\\

\textbf{26:}
\textbf{Mar 17 (Fri)}&
\textbf{Where Did the Grownups Go?  Rebellion and the Generation Gap}\\
&Film Discussion:  \textit{The Graduate}\\
\\

\textbf{27:}
\textbf{Mar 20 (Mon)}&
\textbf{Midterm Exam Review}\\
\\

\textbf{28:}
\textbf{Mar 22 (Wed)}&
\textbf{Midterm Exam}\\
%&\textbf{Annotated Working Bibliography Assignment Due}\\
\\

\textbf{29:}
\textbf{Mar 24 (Fri)}&
\textbf{No Class:  Head Start for Spring Break}\\
\\

\textbf{March 27-29:  Spring Break}\\ 
\\

\textbf{30:}
\textbf{Apr 3 (Mon)}&
\textbf{Where Did the Grownups Go?  Rebellion and the Generation Gap}\\
&Didion, \textit{Some Dreamers of the Golden Dream}\\
\\

\textbf{31:}
\textbf{Apr 5 (Wed)}&
\textbf{Where Did the Grownups Go?  Rebellion and the Generation Gap}\\
&Didion, \textit{On Keeping a Notebook}\\
\\

\textbf{32:}
\textbf{Apr 7 (Fri)}&
\textbf{Where Did the Grownups Go?  Rebellion and the Generation Gap}\\
&Film Discussion:  \textit{Berkeley in the Sixties}\\
&\textbf{Essay 2 Due}\\

\\
\textbf{33:}
\textbf{Apr 10 (Mon)}&
\textbf{The Backlash:  The Emergence of Modern Conservatism}\\
&Perlstein, \textit{Before the Storm}, c. 19\\
&Film Discussion:  Ronald Reagan and \textit{A Time for Choosing}\\
\\

\textbf{34:}
\textbf{Apr 12 (Wed)}&
\textbf{The Backlash:  The Emergence of Modern Conservatism}\\
&Film Discussion:  \textit{The Best of Enemies}\\
\\

\textbf{35:}
\textbf{Apr 14 (Fri)}&
\textbf{No Classes:  Festival of Scholars Event.}\\
&Attendance Required.\\
\\

\textbf{36:}
\textbf{Apr 17 (Mon)}&
\textbf{Cultural Artifacts of the 1960's:  Presentations}\\
\\

\textbf{37:}
\textbf{Apr 19 (Wed)}&
\textbf{Cultural Artifacts of the 1960's:  Presentations}\\
\\

\textbf{38:}
\textbf{Apr 21 (Fri)}&
\textbf{Cultural Artifacts of the 1960's:  Presentations}\\
\\

\textbf{39:}
\textbf{Apr 24 (Mon)}&
\textbf{Cultural Artifacts of the 1960's:  Presentations}\\
\\

\textbf{40:}
\textbf{Apr 26 (Wed)}&
\textbf{Final Exam Review and Wrapup}\\
\\

\textbf{Final Exam}&
\textbf{May 1 (Monday) 1-4pm}\\
&\textbf{Essay Due at the Beginning of Exam Period}\\




\enddata
\end{deluxetable}
\end{document}